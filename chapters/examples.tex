\chapter{Illustrative examples}
\label{chap:examples}

In this chapter we compare, by means of some examples, a number of quasiorders.
Since we considered a large number of preorders on $\Sigma^*$, here we
present only the most interesting comparisons.
Observe that the contribution given by the qos to the algorithm
presented in \cite{ganty2020omegalang} depends if they are used
in the subprocedure \refPrefixes{} or in \refPeriods{}.
In the former case, one qo $\leq_1$ is used only during the computation of
$D_{1,i}^{N_1}(\emptyset)$, where $N_1  \in \mathbb{N}$ is the least value such that
$(D_{1,i}^{N_1+1}(\emptyset))_q \leq_1^{\forall \exists} (D_{1,i}^{N_1}(\emptyset))_q$
for all $q \in Q$ (see Section~\ref{sec:checking-omega-lang-inc}).
Similarly, one qo $\leq_2$ is used only during the computation of
$D_{2,p}^{N_2}(\emptyset)$, where $N_2 \in \mathbb{N}$
the least value such that
$(D_{2,p}^{N_1+1}(\emptyset))_q \leq_2^{\forall \exists} (D_{2,p}^{N_1}(\emptyset))_q$
for all $q \in Q$ and $p \in F$.
For this reason, in the examples we omit the execution of the whole algorithm
and we focus either on the computation of $D_{1,i}^{N_1}(\emptyset)$ or $D_{2,p}^{N_2}(\emptyset)$, depending if
we are considering qos for prefixes or for periods.

Let $\leq$ be a right-monotonic qo on words.
We denote by $\mathscr{A}_{1,i}$ the function that associates $\leq$
with the least $N_1 \in \mathbb{N}$ such that
$(D_{1,i}^{N_1+1}(\emptyset))_q \leq^{\forall \exists} (D_{1,i}^{N_1}(\emptyset))_q$
for all $q \in Q$, i.e. $\mathscr{A}_{1,i}(\leq) = N_1$.
Similarly, we denote by $\mathscr{A}_{2,p}$ the function that associates $\leq$
with the least $N_2 \in \mathbb{N}$ such that
$(D_{2,p}^{N_2+1}(\emptyset))_q \leq^{\forall \exists} (D_{2,p}^{N_2}(\emptyset))_q$
for all $q \in Q$ and for all $p \in F$, i.e. $\mathscr{A}_{2,p}(\leq) = N_2$.
Observe that $\mathscr{A}_{1,i}$ and $\mathscr{A}_{2,p}$ return the number of iterations
of respectively \refPrefixes{} and \refPeriods{}, and hence
we will use them as a metric to compare the qos.

Let $\leq_1$ and $\leq_2$ two right-monotonic qos used for prefixes (periods).
We recall that if $\mathord{\leq_1} \subseteq \mathord{\leq_2}$, then
$\mathord{\mathscr{A}_{1,i}(\leq_2)} \leq \mathord{\mathscr{A}_{1,i}(\leq_1)}$
($\mathord{\mathscr{A}_{2,p}(\leq_2)} \leq \mathord{\mathscr{A}_{2,p}(\leq_1)}$).
We now give practical examples that show that coarser preorders allow the
subprocedures~\refPrefixes{} and~\refPeriods{} to converge in less iterations.

\section{Prefixes}

\begin{figure}[h]
\centering
\begin{tikzpicture}[shorten >=1pt,node distance=2.5cm,auto]
  \tikzstyle{every state}=[fill={rgb:black,1;white,10}]
  \node[state,initial] (q_0) {$q_0$};
  \node[state] (q_2) [above right of=q_0, xshift=2cm, yshift=-0.5cm] {$q_2$};
  \node[state] (q_1) [above of=q_2] {$q_1$};
  \node[state] (q_3) [below right of=q_0, xshift=2cm, yshift=0.5cm] {$q_3$};
  \node[state] (q_5) [below of=q_3] {$q_5$};
  \node[state] (q_4) [left of=q_5] {$q_4$};
  \path[->]
  (q_0) edge node {$a$} (q_1)
  (q_0) edge node {$b$} (q_2)
  (q_2) edge node {$c$} (q_1)
  (q_0) edge node {$a$} (q_3)
  (q_0) edge node {$b$} (q_4)
  (q_4) edge node {$c$} (q_5)
  (q_3) edge [loop right] node {$d$} (q_3)
  (q_5) edge [loop right] node {$d$} (q_5)
  (q_0) edge node {$e$} (q_5)
  ;
\end{tikzpicture}
\caption{Automaton used to compare $\leq^1_{\mathcal{B}}$ and $\wsdir{B}$.}
\label{fig:example-prefix-ctx-vs-wsdir}
\end{figure}

\begin{example}
Let $\mathcal{B}$ be the automaton in
Figure~\ref{fig:example-prefix-ctx-vs-wsdir}.
We compare $\leq^1_{\mathcal{B}}$ with $\wsdir{B}$.
We remark that that $\mathord{\leq^1_{\mathcal{B}}} \subseteq \mathord{\wsdir{B}}$,
because $\forall u,v \in \Sigma^*$,
$\ctx{B}{u} \subseteq \ctx{B}{v} \Longrightarrow$ $\forall (q_1,q_2) \in \ctx{B}{u},
\exists (q_3,q_4) \in \ctx{B}{v}$ such that $q_1 \revdir q_3$ and $q_2 \dir q_4$.
This is due to the fact that $\revdir$ and $\dir$ are reflexive.
Table~\ref{table:D_1-example0} illustrates the first four \emph{Kleene's iterates}
of the function $D_{1,q_0}$.
Consider $\Dine{q_0}{3}{q_1}$.
We observe that $a \leq^1_{\mathcal{B}} bd$ doesn't hold:
$\ctx{B}{a} = \{(q_0,q_1), (q_0,q_3)\}$ and $\ctx{B}{bc} = \{(q_0,q_1), (q_0,q_5)\}$
so that $\ctx{B}{a} \subsetneq \ctx{B}{bc}$.
Since $\exists u \in \Dine{q_0}{3}{q_1}$ such that $\nexists v  \in \Dine{q_0}{2}{q_1}$
such that $v \leq^1_{\mathcal{B}} u$, $\alg{1}{q_0}{\leq^1_{\mathcal{B}}} > 1$.

We observe that $a \wsdir{B} bc$: in fact $q_0 \revdir q_0$, $q_1 \dir q_1$
and $q_3 \dir q_5$.
For this reason, $\forall u \in \Dine{q_0}{3}{q_2}$, $\exists v \in \Dine{q_0}{2}{q_2}$
such that $v \wrfair{B} u$.
The reader can also verify that $a \wsdir{B} ad$, hence
$\forall u \in \Dine{q_0}{3}{q_3}$, $\exists v \in \Dine{q_0}{2}{q_3}$
such that $v \wsdir{B} u$.
Similarly, $e \wsdir{B} bc$ and $e \wsdir{B} ed$, so that
$\forall u \in \Dine{q_0}{3}{q_5}$, $\exists v \in \Dine{q_0}{2}{q_5}$
such that $v \wsdir{B} u$.
This implies that $\alg{1}{q_0}{\wsdir{B}} = 2$.
It is also possible to show that $\alg{1}{q_0}{\leq^1_{\mathcal{B}}} = 3$.
This example shows that for the automaton $\mathcal{B}$,
\texttt{\refPrefixes{}($\mathcal{B}$,$\wsdir{B}$)} converges in strictly less iterations
than \texttt{\refPrefixes{}($\mathcal{B}$,$\leq^1_{\mathcal{B}}$)}.
\end{example}

\begin{table}[h]
\centering
\begin{tabular}{ c | c | c | c | c | c | c }
 & $q_0$ & $q_1$ & $q_2$ & $q_3$ & $q_4$ & $q_5$ \\
\hline
$D_{1,q_0}^0(\emptyset)$ & $\emptyset$ & $\emptyset$ & $\emptyset$ & $\emptyset$ & $\emptyset$ & $\emptyset$ \\
$D_{1,q_0}^1(\emptyset)$ & $\{ \epsilon \}$ & $\emptyset$ & $\emptyset$ & $\emptyset$ & $\emptyset$ & $\emptyset$ \\
$D_{1,q_0}^2(\emptyset)$ & $\{ \epsilon \}$ & $\{a\}$ & $\{b\}$ & $\{a\}$ & $\{b\}$ & $\{e\}$ \\
$D_{1,q_0}^3(\emptyset)$ & $\{ \epsilon \}$ & $\{a,bc\}$ & $\{b\}$ & $\{a,ad\}$ & $\{b\}$ & $\{e, ed, bc\}$ \\
$D_{1,q_0}^4(\emptyset)$ & $\{ \epsilon \}$ & $\{a,bc\}$ & $\{b\}$ & $\{a,ad,add\}$ & $\{b\}$ & $\{e, ed, edd, bc, bcd\}$ \\
\end{tabular}
\caption{The first four Kleene's iterates of $D_{1,q_0}$ on the automaton in
Figure~\ref{fig:example-prefix-ctx-vs-wsdir}}
\label{table:D_1-example0}
\end{table}


\begin{figure}[h]
\centering
\begin{tikzpicture}[shorten >=1pt,node distance=2.5cm,auto]
  \tikzstyle{every state}=[fill={rgb:black,1;white,10}]
  \node[state,initial] (q_0) {$q_0$};
  \node[state] (q_3) [above right of=q_0] {$q_3$};
  \node[state] (q_4) [right of=q_3] {$q_4$};
  \node[state] (q_1) [above of=q_3] {$q_1$};
  \node[state] (q_2) [right of=q_1] {$q_2$};
  \node[state] (q_5) [below right of=q_0, xshift=2.5cm] {$q_5$};
  \node[state] (q_6) [below right of=q_0, yshift=-2.5cm] {$q_6$};
  \node[state] (q_7) [right of=q_6] {$q_7$};
  \path[->]
  (q_0) edge node {$a$} (q_1)
  (q_0) edge node {$g$} (q_5)
  (q_0) edge node {$a$} (q_6)
  (q_6) edge node {$b$} (q_7)
  (q_1) edge node {$b$} (q_2)
  (q_2) edge [loop right] node {$c$} (q_2)
  (q_0) edge node {$d$} (q_3)
  (q_3) edge node {$e$} (q_4)
  (q_4) edge [right] node {$f$} (q_2)
  (q_4) edge node {$f$} (q_5)
  (q_5) edge [loop right] node {$c$} (q_5)
  ;
\end{tikzpicture}
\caption{Automaton used to compare $\wrs{B}$ and $\wrdir{B}$.}
\label{fig:example-prefix-state-vs-right-sim}
\end{figure}

\begin{example}
Let $\mathcal{B}$ be the automaton in Figure~\ref{fig:example-prefix-state-vs-right-sim}.
We compare $\wrs{B}$ with $\wrdir{B}$.
We remark that $\mathord{\wrs{B}} \subseteq \mathord{\wrdir{B}}$,
because $\forall u,v \in \Sigma^*$,
$post_u^{\mathcal{B}}(I) \subseteq post_v^{\mathcal{B}}(I) \Longrightarrow
\forall q \in post_u^{\mathcal{B}}(I) \; \exists p \in post_v^{\mathcal{B}}(I)$
such that $q \dir p$.
This is due to the fact that $\dir$ is reflexive.
Table~\ref{table:D_1-example1} illustrates the first five \emph{Kleene's iterates}
of the function $D_{1,q_0}$.
Consider $(D_{1,q_0}^4(\emptyset))_{q_2}$.
Observe that $post_{def}^{\mathcal{B}}(I) = \{q_2, q_5\}$, while
$post_{ab}^{\mathcal{B}}(I) = \{q_2, q_7\}$, so that $ab \wrs{B} def$ doesn't hold.
For this reason, $\nexists u \in (D_{1,q_0}^3(\emptyset))_{q_2}$ such that $u \wrs{B} def$.
Observing that $def \in (D_{1,q_0}^4(\emptyset))_{q_2}$, we conclude that
$\mathscr{A}_{1,q_0}(\wrs{B}) > 3$.

On the other hand, $\forall p \in post_{ab}^{\mathcal{B}}(I)$,
$\exists q \in post_{def}^{\mathcal{B}}(I)$ such that $p \dir q$:
in fact $q_2 \dir q_2$ and $q_7 \dir q_2$.
This implies $ab \dir def$, and then $\forall u \in (D_{1,q_0}^4(\emptyset))_{q_2}$,
$\exists v \in (D_{1,q_0}^3 (\emptyset))_{q_2}$ such that $v \wrdir{B} u$.
The reader can also verify that $\forall u \in (D_{1,q_0}^4(\emptyset))_{q_5}$,
$\exists v \in (D_{1,q_0}^3 (\emptyset))_{q_5}$ such that $v \wrdir{B} u$.
This implies that $\mathscr{A}_{1,q_0}(\wrdir{B}) = 3$.
It is also possible to show that $\mathscr{A}_{1,q_0}(\wrs{B}) = 4$.
This example shows that \texttt{\refPrefixes{}($\mathcal{B}$,$\wrdir{B}$)} converges in
strictly less iterations than \texttt{\refPrefixes{}($\mathcal{B}$,$\wrs{B}$)}.
\end{example}

\begin{table}[h]
\centering
\begin{tabular}{ c | c | c | c | c | c | c | c | c}
 & $q_0$ & $q_1$ & $q_2$ & $q_3$ & $q_4$ & $q_5$ & $q_6$ & $q_7$\\
\hline
$D_{1,q_0}^0(\emptyset)$ & $\emptyset$ & $\emptyset$ & $\emptyset$ & $\emptyset$ & $\emptyset$ & $\emptyset$ & $\emptyset$ & $\emptyset$\\
$D_{1,q_0}^1(\emptyset)$ & $\{ \epsilon \}$ & $\emptyset$ & $\emptyset$ & $\emptyset$ & $\emptyset$ & $\emptyset$ & $\emptyset$ & $\emptyset$\\
$D_{1,q_0}^2(\emptyset)$ & $\{ \epsilon \}$ & $\{a\}$ & $\emptyset$ & $\{d\}$ & $\emptyset$ & $\{g\}$ & $\{a\}$ & $\emptyset$\\
$D_{1,q_0}^3(\emptyset)$ & $\{ \epsilon \}$ & $\{a\}$ & $\{ab\}$ & $\{d\}$ & $\{de\}$ & $\{g, gc\}$ & $\{a\}$ & $\{ab\}$\\
$D_{1,q_0}^4(\emptyset)$ & $\{ \epsilon \}$ & $\{a\}$ & $\{ab, abc, def\}$ & $\{d\}$ & $\{de\}$ & $\{g, gc, gcc, def\}$ & $\{a\}$ & $\{ab\}$\\
$D_{1,q_0}^5(\emptyset)$ & $\{ \epsilon \}$ & $\{a\}$ & $\{ab, abc, abcc, def\}$ & $\{d\}$ & $\{de\}$ & $\{g, gc, gcc, gccc, def\}$ & $\{a\}$ & $\{ab\}$\\
\end{tabular}
\caption{The first five Kleene's iterates of $D_{1,q_0}$ on the automaton in
Figure~\ref{fig:example-prefix-state-vs-right-sim}}
\label{table:D_1-example1}
\end{table}

% Per lesempio di delayed vs direct on words basta rendere q_5 finale
\begin{figure}[h]
\centering
\begin{tikzpicture}[shorten >=1pt,node distance=2.5cm,auto]
  \tikzstyle{every state}=[fill={rgb:black,1;white,10}]
  \node[state,initial] (q_0) {$q_0$};
  \node[state] (q_2) [below of=q_0] {$q_2$};
  \node[state] (q_1) [left of=q_2] {$q_1$};
  \node[state,accepting] (q_3) [right of=q_2] {$q_3$};
  \path[->]
  (q_0) edge node {$a$} (q_1)
  (q_0) edge node {$c$} (q_2)
  (q_0) edge node {$c$} (q_3)
  (q_1) edge node {$b$} (q_2)
  (q_2) edge node {$d$} (q_3)
  (q_3) edge [loop below] node {$d$} (q_3)
  ;
\end{tikzpicture}
\caption{Automaton used to compare $\wrdir{B}$ and $\wrdel{B}$.}
\label{fig:example-prefix-right-sim-vs-right-del-sim}
\end{figure}

\begin{example}
Let $\mathcal{B}$ be the automaton in
Figure~\ref{fig:example-prefix-right-sim-vs-right-del-sim}.
We compare $\wrdir{B}$ with $\wrdel{B}$.
We remark that $\mathord{\wrdir{B}} \subseteq \mathord{\wrdel{B}}$,
because $\forall u,v \in \Sigma^*$,
if $\forall p \in post_u^{\mathcal{B}}(I), \exists q \in post_v^{\mathcal{B}}(I)$
such that $p \dir q$, then it also holds that $p \del q$.
This is due to the fact that $\mathord{\dir} \subseteq \mathord{\del}$
(see Section~\ref{sec:simulation}).
Table~\ref{table:D_1-example2} illustrates the first four \emph{Kleene's iterates}
of the function $D_{1,q_0}$.
Consider $\Dine{q_0}{3}{q_2}$.
We observe that $c \wrdir{B} ab$ doesn't hold:
$\posti{c}{B} = \{q_2, q_3\}$, $\posti{ab}{B} = \{q_2\}$ and since $q_2$ is
not a final state $q_3 \dir q_2$ doesn't hold.
This implies that $\alg{1}{q_0}{\wrdir{B}} > 2$.

We observe that $c \wrdel{B} ab$: in fact $q_3 \del q_2$.
From $q_3$ starts the infinite (and fair)
trace $\pi_0 = q_3 \trans{d} q_3 \trans{d} \cdots$ in which each state is final.
From $q_2$ starts the infinite (and fair) trace
$\pi_1 = q_2 \trans{d} q_3 \trans{d} \cdots$ that matches $\pi_0$.
In particular, for each final state in $\pi_0$ there exists a corresponding
final state in $\pi_1$.

It also holds that $c \wrdel{B} cd$: $\posti{c}{B} = \{q_2,q_3\}$,
$\posti{cd}{B} = \{q_3\}$, $q_2 \del q_3$ and $q_3 \del q_3$.
This implies that $\alg{1}{q_0}{\wrdel{B}} = 2$.
It is also possible to show that $\alg{1}{q_0}{\wrdir{B}} = 3$.
This example shows that for the automaton $\mathcal{B}$,
\texttt{\refPrefixes{}($\mathcal{B}$,$\wrdel{B}$)} converges in strictly less iterations
than \texttt{\refPrefixes{}($\mathcal{B}$,$\wrdir{B}$)}.
\end{example}

\begin{table}[h]
\centering
\begin{tabular}{ c | c | c | c | c }
 & $q_0$ & $q_1$ & $q_2$ & $q_3$ \\
\hline
$D_{1,q_0}^0(\emptyset)$ & $\emptyset$ & $\emptyset$ & $\emptyset$ & $\emptyset$ \\
$D_{1,q_0}^1(\emptyset)$ & $\{\epsilon\}$ & $\emptyset$ & $\emptyset$ & $\emptyset$ \\
$D_{1,q_0}^2(\emptyset)$ & $\{\epsilon\}$ & $\{a\}$ & $\{c\}$ & $\{c\}$ \\
$D_{1,q_0}^3(\emptyset)$ & $\{\epsilon\}$ & $\{a\}$ & $\{c, ab\}$ & $\{c, cd\}$ \\
$D_{1,q_0}^4(\emptyset)$ & $\{\epsilon\}$ & $\{a\}$ & $\{c, ab\}$ & $\{c, cd, cdd\}$ \\
\end{tabular}
\caption{The first five Kleene's iterates of $D_{1,q_0}$ on the automaton in
Figure~\ref{fig:example-prefix-right-sim-vs-right-del-sim}}
\label{table:D_1-example2}
\end{table}


\begin{figure}[h]
\centering
\begin{tikzpicture}[shorten >=1pt,node distance=2.5cm,auto]
  \tikzstyle{every state}=[fill={rgb:black,1;white,10}]
  \node[state,initial] (q_0) {$q_0$};
  \node[state] (q_3) [above right of=q_0, yshift=-0.5cm] {$q_3$};
  \node[state] (q_4) [right of=q_3] {$q_4$};
  \node[state] (q_1) [above of=q_3] {$q_1$};
  \node[state] (q_2) [right of=q_1] {$q_2$};
  \node[state] (q_7) [below right of=q_0, xshift=2.5cm, yshift=0.5cm] {$q_7$};
  \node[state,accepting] (q_6) [below of=q_7] {$q_6$};
  \node[state] (q_5) [left of=q_6] {$q_5$};
  \path[->]
  (q_0) edge node {$a$} (q_1)
  (q_1) edge node {$b$} (q_2)
  (q_2) edge [loop right] node {$c$} (q_2)
  (q_0) edge node {$d$} (q_3)
  (q_3) edge node {$e$} (q_4)
  (q_4) edge node {$f$} (q_2)
  (q_4) edge node {$f$} (q_7)
  (q_0) edge node {$a$} (q_5)
  (q_5) edge node {$b$} (q_6)
  (q_6) edge node {$a$} (q_7)
  (q_0) edge node {$g$} (q_7)
  (q_7) edge [loop right] node {$a$} (q_7)
  ;
\end{tikzpicture}
\caption{Automaton used to compare $\wrdel{B}$ and $\wrfair{B}$.}
\label{fig:example-prefix-del-vs-fair}
\end{figure}

\begin{example}
Let $\mathcal{B}$ be the automaton in
Figure~\ref{fig:example-prefix-del-vs-fair}.
We compare $\wrdel{B}$ with $\wrfair{B}$.
We remark that $\mathord{\wrdel{B}} \subseteq \mathord{\wrfair{B}}$,
because $\forall u,v \in \Sigma^*$,
if $\forall p \in post_u^{\mathcal{B}}(I), \exists q \in post_v^{\mathcal{B}}(I)$
such that $p \del q$, then it also holds that $p \fair q$.
This is due to the fact that $\mathord{\del} \subseteq \mathord{\fair}$
(see Section~\ref{sec:simulation}).
Table~\ref{table:D_1-example3} illustrates the first four \emph{Kleene's iterates}
of the function $D_{1,q_0}$.
Consider $\Dine{q_0}{3}{q_2}$.
We observe that $ab \wrdel{B} def$ doesn't hold:
$\posti{ab}{B} = \{q_2, q_6\}$, $\posti{def}{B} = \{q_2, q_7\}$
and while $q_2 \del q_2$, it is not true that $q_6 \del q_2$ or $q_6 \del q_7$.
This is due to the fact that the trace $\pi_0 = q_6 \trans{a} q_7 \trans{a} \cdots$
contains a final state, and the only trace $\pi_1 = q_7 \trans{a} q_7 \trans{a} \cdots$
that starts from $q_7$ that can match the $a$'s contains no final state.
This implies that $\alg{1}{q_0}{\wrdel{B}} > 2$.

We observe that $ab \wrfair{B} def$: in fact $q_6 \fair q_7$.
From $q_6$ starts the trace $\pi_0$, but since it doesn't contain one infinite
number of final states it is not fair.
This implies that since $q_7$ can match $\pi_0$ with $\pi_1$, $q_6 \fair q_7$.
For this reason, $ab \wrfair{B} def$ so that
$\forall u \in \Dine{q_0}{3}{q_2}$, $\exists v \in \Dine{q_0}{2}{q_2}$
such that $v \wrfair{B} u$.
The reader can also verify that $g \wrfair{B} gaa$, $g \wrfair{B} aba$
and $g \wrfair{B} def$, hence
$\forall u \in \Dine{q_0}{3}{q_7}$, $\exists v \in \Dine{q_0}{2}{q_7}$
such that $v \wrfair{B} u$.
This implies that $\alg{1}{q_0}{\wrfair{B}} = 2$.
It is also possible to show that $\alg{1}{q_0}{\wrdel{B}} = 3$.
This example shows that for the automaton $\mathcal{B}$,
\texttt{\refPrefixes{}($\mathcal{B}$,$\wrfair{B}$)} converges in strictly less iterations
than \texttt{\refPrefixes{}($\mathcal{B}$,$\wrdel{B}$)}.
\end{example}

\begin{table}[h]
\centering
\begin{tabular}{ c | c | c | c | c | c | c | c | c}
 & $q_0$ & $q_1$ & $q_2$ & $q_3$ & $q_4$ & $q_5$ & $q_6$ & $q_7$\\
\hline
$D_{1,q_0}^0(\emptyset)$ & $\emptyset$ & $\emptyset$ & $\emptyset$ & $\emptyset$ & $\emptyset$ & $\emptyset$ & $\emptyset$ & $\emptyset$\\
$D_{1,q_0}^1(\emptyset)$ & $\{ \epsilon \}$ & $\emptyset$ & $\emptyset$ & $\emptyset$ & $\emptyset$ & $\emptyset$ & $\emptyset$ & $\emptyset$\\
$D_{1,q_0}^2(\emptyset)$ & $\{ \epsilon \}$ & $\{a\}$ & $\emptyset$ & $\{d\}$ & $\emptyset$ & $\{a\}$ & $\emptyset$ & $\{g\}$\\
$D_{1,q_0}^2(\emptyset)$ & $\{ \epsilon \}$ & $\{a\}$ & $\{ab\}$ & $\{d\}$ & $\{de\}$ & $\{a\}$ & $\{ab\}$ & $\{g, ga\}$\\
$D_{1,q_0}^3(\emptyset)$ & $\{ \epsilon \}$ & $\{a\}$ & $\{ab, abc, def\}$ & $\{d\}$ & $\{de\}$ & $\{a\}$ & $\{ab\}$ & $\{g, ga, gaa, aba, def\}$\\
$D_{1,q_0}^4(\emptyset)$ & $\{ \epsilon \}$ & $\{a\}$ & $\{ab, abc, abcc,$ & $\{d\}$ & $\{de\}$ & $\{a\}$ & $\{ab\}$ & $\{g, ga, gaa, gaaa,$\\
&  &  & $def, defc\}$ &  &  &  &  & $aba, abaa, def, defa\}$\\
\end{tabular}
\caption{The first four Kleene's iterates of $D_{1,q_0}$ on the automaton in
Figure~\ref{fig:example-prefix-del-vs-fair}}
\label{table:D_1-example3}
\end{table}


\section{Periods}

\begin{figure}[h]
\centering
\begin{tikzpicture}[shorten >=1pt,node distance=2.5cm,auto]
  \tikzstyle{every state}=[fill={rgb:black,1;white,10}]
  \node[state,initial] (q_0) {$q_0$};
  \node[state] (q_1) [above right of=q_0, xshift=1.25cm] {$q_1$};
  \node[state] (q_2) [right of=q_1] {$q_2$};
  \node[state] (q_3) [below right of=q_0] {$q_3$};
  \node[state] (q_4) [right of=q_3] {$q_4$};
  \node[state] (q_5) [right of=q_4] {$q_5$};
  \node[state,accepting] (q_6) [above right of=q_5] {$q_6$};
  \node[state,accepting] (q_7) [above right of=q_6] {$q_7$};
  \node[state] (q_8) [below right of=q_6] {$q_8$};
  \path[->]
  (q_0) edge node {$a$} (q_1)
  (q_1) edge node {$b$} (q_2)
  (q_2) edge node {$c$} (q_6)
  (q_0) edge node {$a,c$} (q_3)
  (q_3) edge node {$d$} (q_4)
  (q_4) edge node {$e$} (q_5)
  (q_5) edge node {$a,c$} (q_6)
  (q_6) edge node {$b$} (q_7)
  (q_6) edge node {$d$} (q_8)
  (q_8) edge node {$e$} (q_7)
  (q_7) edge [loop above] node {$a$} (q_7)
  ;
\end{tikzpicture}
\caption{Automaton used to compare $\wssf{B}$ and $\wsdirf{B}$.}
\label{fig:example-state-vs-dir-periods}
\end{figure}

\begin{example}
Let $\mathcal{B}$ be the automaton in
Figure~\ref{fig:example-state-vs-dir-periods}.
We compare $\wssf{B}$ with $\wsdirf{B}$.
We remark that $\mathord{\wssf{B}} \subseteq \mathord{\wsdirf{B}}$,
because $\forall u,v \in \Sigma^*$,
if $\ctx{B}{u} \subseteq \ctx{B}{v}$ and $\ctxf{B}{u} \subseteq \ctxf{B}{v}$ then
$\forall (q_1,q_2) \in \ctx{B}{u}$ $\exists (q_3,q_4) \in \ctx{B}{v}$
such that $q_1 \bwdir q_3$ and $q_2 \dir q_4$.
In particular, consider $q_3 = q_1$ and $q_4 = q_2$.
Furthermore, $q_1 \goesf{u} q_2$ implies $(q_1,q_2) \in \ctxf{B}{u}$,
so that $(q_3,q_4) \in \ctxf{B}{v}$.
Table~\ref{table:D_2-example1} illustrates the first three \emph{Kleene's iterates}
of the function $\Dp{q_6}$.
Consider $\Dpne{q_6}{2}{q_7}$.
We observe that $b \wssf{B} de$ doesn't hold:
$\ctx{B}{b} = \{(q_1,q_2), (q_6,q_7)\}$ and $\ctx{B}{de} = \{(q_3,q_5), (q_6,q_7)\}$.
For this reason, $\exists u \in \Dpne{q_6}{2}{q_7}$ such that
$\nexists v \in \Dpne{q_6}{1}{q_7}$ such that $v \wssf{B} u$.
This implies that $\alg{2}{q_6}{\wssf{B}} > 1$.

We observe that $b \wsdirf{B} de$.
This is due to the fact that $q_1 \bwdir q_3$, $q_2 \dir q_5$, $q_6 \bwdir q_6$
and $q_7 \dir q_7$.
For this reason,
$\forall u \in \Dpne{q_6}{2}{q_7}$, $\exists v \in \Dpne{q_6}{1}{q_7}$
such that $v \wsdirf{B} u$.
This implies that $\alg{2}{q_6}{\wsdirf{B}} = 1$.
It is also possible to show that $\alg{2}{q_6}{\wssf{B}} = 2$.
This example shows that for the automaton $\mathcal{B}$,
\texttt{\refPeriods{}($\mathcal{B}$, $\wssf{B}$)} converges in strictly less iterations
than \texttt{\refPeriods{}($\mathcal{B}$,$\wsdirf{B}$)}.
\end{example}

\begin{table}[h]
\centering
\begin{tabular}{ c | c | c | c | c | c | c | c | c | c}
 & $q_0$ & $q_1$ & $q_2$ & $q_3$ & $q_4$ & $q_5$ & $q_6$ & $q_7$ & $q_8$\\
\hline
$\Dpn{q_6}{0}$ & $\emptyset$ & $\emptyset$ & $\emptyset$ & $\emptyset$ & $\emptyset$ & $\emptyset$ & $\emptyset$ & $\emptyset$ & $\emptyset$\\
$\Dpn{q_6}{1}$ & $\emptyset$ & $\emptyset$ & $\emptyset$ & $\emptyset$ & $\emptyset$ & $\emptyset$ & $\emptyset$ & $\{b\}$ & $\{d\}$\\
$\Dpn{q_6}{2}$ & $\emptyset$ & $\emptyset$ & $\emptyset$ & $\emptyset$ & $\emptyset$ & $\emptyset$ & $\emptyset$ & $\{b,ba,de\}$ & $\{d\}$\\
$\Dpn{q_6}{3}$ & $\emptyset$ & $\emptyset$ & $\emptyset$ & $\emptyset$ & $\emptyset$ & $\emptyset$ & $\emptyset$ & $\{b,ba,baa,de,dea\}$ & $\{d\}$\\
\end{tabular}
\caption{The first three Kleene's iterates of $\Dp{q_6}$ on the automaton in
Figure~\ref{fig:example-state-vs-dir-periods}}
\label{table:D_2-example1}
\end{table}

